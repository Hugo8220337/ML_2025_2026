\section{Interface de Utilização} \label{sec:interface-utilizador}

De forma a cumprir o requisito de desenvolver uma aplicação que permita a interação com o modelo treinado, foi criada uma interface \textit{web}, com o objetivo de permitir que qualquer utilizador verifique a veracidade de uma notícia e, simultaneamente, oferecer uma visão transparente sobre a performance dos modelos desenvolvidos.

A aplicação foi desenvolvida na linguagem Python (biblioteca \textit{Streamlit}\footnote{https://streamlit.io}) e encontra-se dividida em dois módulos principais.

\subsection{Deteção em Tempo Real} \label{subsec:detecao-tempo-real}

Este é o módulo principal, onde o utilizador interage com o sistema final. O fluxo de utilização resume-se a três passos:

\begin{enumerate}
	\item \textbf{Inserção do URL:} O utilizador introduz o \textit{link} da notícia que deseja verificar;
	\item \textbf{Extração Automática:} O sistema executa um algoritmo de \textit{web scraping} que acede à página e extrai automaticamente o título e o corpo da notícia;
	\item \textbf{Classificação:} O texto extraído é processado pelos modelos especializados e pelo Meta-Classificador, devolvendo o resultado final (Verdadeiro/Falso) e o nível de confiança.
\end{enumerate}

\begin{figure}[h]
	\centering
	\includegraphics[width=0.7\linewidth]{imagens/user_interface_main}
	\caption{Interface de deteção de \textit{Fake News}}
	\label{fig:interface-exemplo}
\end{figure}

\subsection{Painel de Visualizações e Métricas} \label{subsec:visualizacao-metricas}

Para garantir que os resultados do projeto são transparentes e fáceis de analisar, foi adicionado um separador dedicado a "Visualizações". Nesta área, o utilizador pode consultar graficamente o desempenho de todos os modelos criados durante o projeto.

O painel inclui:
\begin{itemize}
	\item \textbf{Comparação de Modelos:} Gráficos de barras que mostram os F1-Scores de todos os algoritmos testados, permitindo perceber rapidamente qual foi o "vencedor" em cada categoria;
	\item \textbf{Distribuição de Dados:} Gráficos de dispersão (\textit{scatterplots}) que ilustram como os modelos separam as notícias verdadeiras das falsas.
\end{itemize}

\begin{figure}[h]
	\centering
	\includegraphics[width=0.8\linewidth]{imagens/user_interface_visualizations}
	\caption{Painel de métricas da aplicação: Comparação do desempenho dos modelos (F1-Score) e visualização da distribuição das classes.}
	\label{fig:dashboard-visualizacoes}
\end{figure}
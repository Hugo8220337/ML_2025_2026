\section{Desenvolvimento dos Modelos (Metodologia)} \label{sec:desenvolvimento-modelos}
% Aqui descreves o "coração" do trabalho. Para cada sub-tópico abaixo, deves identificar e justificar os algoritmos e parâmetros.

O presente capítulo detalha a metodologia adotada para o desenvolvimento do sistema de deteção de \textit{Fake News}. Dada a natureza multidimensional da desinformação, optou-se por uma arquitetura modular hierárquica (abordagem inspirada em \textit{Stacking Ensemble}), em vez de um único modelo monolítico.

Para tal, foram desenvolvidos modelos especialistas independentes, treinados em \textit{datasets} distintos, cujo objetivo é capturar diferentes nuances linguísticas e estruturais das notícias. As saídas probabilísticas destes modelos funcionam como \textit{features} de alto nível (meta-features) para o classificador final.

A arquitetura proposta compreende os seguintes módulos:

\begin{itemize}
	\item \textbf{Classificação de Tópicos:} Contextualização temática do artigo (ex: Política, Saúde, Tecnologia);
	\item \textbf{Análise de Anomalias:} Identificação de padrões nos textuais em notícias verdadeiras de modo a detetar anomalias;
	\item \textbf{Deteção de Stance (Postura):} Análise da concordância entre o título e o corpo da notícia;
	\item \textbf{Deteção de Clickbait:} Análise de padrões sensacionalistas nos títulos;
	\item \textbf{Meta-Classificador (Modelo Final):} Agregação das saídas anteriores para a previsão final de veracidade.
\end{itemize}

Nas subsecções seguintes, é descrito o ciclo de vida de desenvolvimento para cada um destes componentes, abrangendo desde o pré-processamento específico e engenharia de atributos (\textit{Feature Engineering}), até à justificação da escolha dos algoritmos.

% Para cada modelo devo falar dos algoritmos testados e a justificação da escolha
\subsection{Modelo 1: Classificação de Tópicos} \label{subsec:topic-classification}

\subsection{Modelo 2: Análise de Anomalias} \label{subsec:anomaly-analysis}

\subsection{Modelo 3: Deteção de Stance (Postura)} \label{subsec:stance-detection}

\subsection{Modelo 4: Deteção de Clickbait} \label{subsec:clickbait-detection}

\subsection{Modelo Final: Fake News Meta-Classifier} \label{subsec:modelo-final}
% Como as saídas dos modelos anteriores (1, 2 e 3) entram neste modelo.